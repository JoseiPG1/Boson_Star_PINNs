\documentclass{article}
\usepackage[utf8]{inputenc}
\usepackage{amsmath}
\usepackage{amsfonts}
\usepackage{cite} 
\usepackage{breqn}
\usepackage{graphicx}
\usepackage{hyperref}

\title{Numerical solutions for boson stars in $f\left(\mathcal{R}\right)$ gravity using PINNs}
\author{Jos\'e Perdiguero G\'arate}

\begin{document}

\maketitle
%\tableofcontents

\section{Introduction}
\label{sec:introduction}

\section{Dynamics in $f\left(\mathcal{R}\right)$}
\label{sec:dynamics}

In Palatini's formalisms, the manifold is endowed with two fundamental and independent
fields: the metric tensor $g_{\mu\nu}$ and the affine connection $\Gamma^{\alpha}{}_{\beta\gamma}$, 
the former allow us to define the notion of distance, whereas the latter define the notion of 
parallelism. The action is written as follow
\begin{equation}
S\left[g, \Gamma\right] = \frac{1}{2k}\int \mathrm{d}^{4}x \sqrt{-g}f\left(\mathcal{R}\right) 
    - \frac{1}{2}\int \mathrm{d}^{4}x \sqrt{-g}\mathcal{P}(X, \Phi),
\end{equation}
where the first term correspond to a generalization of the Einstein-Hilbert action, by replacing
the functional with an arbitrary function of the Ricci scalar, and the second term stands for the
matter sector defined as
\begin{equation}
\mathcal{P}  = X - 2V(\Phi),
\end{equation}
where $X = g^{\alpha\beta}\partial_\alpha {\Phi} \partial_\beta \Phi$ and $V(\Phi) = -\frac{1}{2}\mu^2\Phi\Phi$,
with $\mu$ as the mass of the complex scalar field. The field equations are obtained through varying
the action with respect to the inverse metric tensor $g^{\mu\nu}$ and the affine connection
$\Gamma^{\alpha}{}_{\beta\gamma}$ respectively


\section{Numerical analysis}
\label{sec:numercal_analysis}

\section{Final remarks}
\label{sec:remarks}

\end{document}