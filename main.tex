\documentclass{article}
\usepackage[utf8]{inputenc}
\usepackage{amsmath}
\usepackage{amsfonts}
\usepackage{cite} 
\usepackage{breqn}
\usepackage{graphicx}
\usepackage{hyperref}

\title{Numerical solutions for boson stars in $f\left(\mathcal{R}\right)$ gravity using PINNs}
\author{Jos\'e Perdiguero G\'arate}

\begin{document}

\maketitle
%\tableofcontents

\section{Introduction}
\label{sec:introduction}

\section{Dynamics in $f\left(\mathcal{R}\right)$}
\label{sec:dynamics}

In Palatini's formalisms, the manifold is endowed with two fundamental and independent
fields: the metric tensor $g_{\mu\nu}$ and the affine connection $\Gamma^{\alpha}{}_{\beta\gamma}$, 
the former allow us to define the notion of distance, whereas the latter define the notion of 
parallelism. The action is written as follow
\begin{equation}
    S\left[g, \Gamma\right] = \frac{1}{2k}\int \mathrm{d}^{4}x \sqrt{-g}f\left(\mathcal{R}\right) 
    - \frac{1}{2}\int \mathrm{d}^{4}x \sqrt{-g}\mathcal{P}(X, \Phi),
\end{equation}
where the first term correspond to a generalization of the Einstein-Hilbert action, by replacing
the functional with an arbitrary function of the Ricci scalar, and the second term stands for the
matter sector defined as
\begin{equation}
    \mathcal{P}  = X - 2V(\Phi),
\end{equation}
where $X = g^{\alpha\beta}\partial_\alpha {\Phi} \partial_\beta \Phi$ and $V(\Phi) = -\frac{1}{2}\mu^2\Phi\Phi$,
with $\mu$ as the mass of the complex scalar field. The field equations are obtained through varying
the action with respect the fundamental fields
\begin{equation}
\label{g_feq}
    f_{\mathcal{R}}\mathcal{R}_{\mu\nu} - \frac{1}{2}g_{\mu\nu}f_{\mathcal{R}}
    + \left(g_{\mu\nu}\square - \nabla_\mu\nabla_\nu\right)f_{\mathcal{R}} = k \mathcal{T}_{\mu\nu},
\end{equation}
\begin{equation}
\label{cg_feq}
    \nabla_\lambda\left(\sqrt{-g}f_{\mathcal{R}} g^{\mu\nu}\right) = 0
\end{equation}
where Eqs.\eqref{g_feq} and \eqref{cg_feq} are the field equation coming from varying the action with respect
to the inverse metric tensor $g^{\mu\nu}$ and the affine connection $\Gamma^{\alpha}{}_{\beta\gamma}$
respectively. Additionally, the energy-momentum tensor is defined as
\begin{equation}
    \mathcal{T}_{\mu\nu} = -\frac{2}{\sqrt{-g}} \frac{\delta\left(\sqrt{-g}\mathcal{P}(X, \Phi)\right)}{\delta g^{\mu\nu}}.
\end{equation}

An exact solution to Eq.\eqref{cg_feq} can be found by introducing a conformal transformation
of the metric tensor
\begin{equation}
    q_{\mu\nu} = f_{\mathcal{R}}g_{\mu\nu}.
\end{equation}
where the conformal factor is written as $f_{\mathcal{R}}$. The conformal transformation, allow us
to write an explicit relation between the affine connection with the conformal metric
\begin{equation}
    \Gamma^{\lambda}{}_{\mu\nu} = \frac{1}{2}q^{\lambda\rho}\left(\partial_\mu q_{\rho\nu} + 
    \partial_\nu q_{\rho\mu} - \partial_\rho q_{\mu\nu}\right),
\end{equation}
this are the Christoffel symbols, and we can use it in Eq.\eqref{g_feq} to find an equation
that only involves the metric tensor and the matter.
\begin{equation}
    \mathcal{R}_{\mu\nu} - \frac{1}{2}q_{\mu\nu}\mathcal{R} = \frac{k}{f_{\mathcal{R}}}\mathcal{T}_{\mu\nu}
    - \frac{\mathcal{R}f_{\mathcal{R}} - f}{2f_{\mathcal{R}}}q_{\mu\nu} - \frac{3}{2f^2_{\mathcal{R}}}
    \left(\partial_\mu f_{\mathcal{R}} \partial_\nu f_{\mathcal{R}} - \frac{1}{2}q_{\mu\nu}(\partial f_{\mathcal{R}})^2\right)
    + \frac{1}{f_{\mathcal{R}}}\left(\nabla_\mu\nabla_\nu f_{\mathcal{R}} - q_{\mu\nu}\square f_{\mathcal{R}}\right)
\end{equation}

For the sake of simplicity we shall consider the following form of the functional
\begin{equation}
    f(\mathcal{R}) = \mathcal{R} + \xi \mathcal{R}^2.
\end{equation}
The above functional correspond to a quadratic correction to the Ricci scalar, which has
its origin at the level of quantum field theory. It was first used by Starobinsky in order 
to explain inflation.

Working in the Einstein-frame, the action is written as
\begin{equation}
    S = \frac{1}{2k}\int \mathrm{d}^4x \sqrt{-q}\mathcal{R} - \frac{1}{2}\int \mathrm{d}^4x \sqrt{-q}\mathcal{K}(Z,\Phi),
\end{equation}
where $\mathcal{K}(Z,\Phi)$ is defined as
\begin{equation}
    \mathcal{K}(Z,\Phi) = \frac{Z - \xi k Z^2}{1 - 8\xi k V} - \frac{2V}{1 - 8\xi k V}.
\end{equation}
The first field equation, is obtained by varying the action with respect to the
inverse metric $g^{\mu\nu}$
\begin{equation}
    \mathcal{R}_{\mu\nu} - \frac{1}{2}\mathcal{R}g_{\mu\nu} = k\mathcal{T}_{\mu\nu},
    \end{equation} 
where the energy-momentum tensor $\mathcal{T}_{\mu\nu}$ is defined as
\begin{equation}
    \mathcal{T}_{\mu\nu} = \frac{\left(\partial_\mu \Phi \partial_\nu \Phi + \partial_\mu \Phi \partial_\nu \Phi \right)\left(1 - 2\xi k Z\right)
    -q_{\mu\nu}\left( Z\left(1 - \xi k Z\right) +  \mu^2\vert \Phi\vert^2\right)}{2 + 8\xi k \mu^2 \vert \Phi\vert^2}.
\end{equation}
The second field equation is obtained with the variation with respect to the complex
scalar field 

\begin{equation}
    \mathrm{d}s^2 = -\alpha^2(x) \mathrm{d}t^2 + \beta^2(x)\mathrm{d}x^2
    + x^2\mathrm{d}\theta^2 + x^2\sin^2\theta\mathrm{d}\varphi^2
\end{equation}



\section{Numerical analysis}
\label{sec:numercal_analysis}

\section{Final remarks}
\label{sec:remarks}

\end{document}